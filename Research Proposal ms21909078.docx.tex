\documentclass[12pt,a4paper]{article}
\usepackage[latin2]{inputenc}
\usepackage{graphicx}
\usepackage{ulem}
\usepackage{amsmath}
\begin{document}
\begin{center}\textbf{DOCKER IMAGE SECURITY FOR DEVSECOPS}
\end{center}



\begin{center}\textbf{RESEARCH PROPOSAL}\end{center}



\begin{center}\textbf{M.Sc IN INFORMATION TECHNOLOGY (CYBER SECURITY)
}\end{center}

\begin{center}FACULTY OF GRADUATE STUDIES \& RESEARCH\end{center}





\begin{center}\textbf{SUBMITTED BY}\end{center}

\begin{center}CSK Pathirana (MS21909078)\end{center}



\begin{center}Feb 2021\end{center}







\begin{center}\textbf{LECTURER}\end{center}

\begin{center}Dr. Lakmal Rupasinghe\end{center}



\section{\begin{center}SRI LANKA INSTITUTE OF INFORMATION TECHNOLOGY
\end{center}}


\newpage
\underline{PROBLEM DEFINITION}



\newcounter{numberedCntB}
\begin{enumerate}
\item Docker containers are private registry servers which wrap a piece 
of software in a complete file system that contains everything it needs 
to run: code, runtime, system tools, system libraries and anything that 
can install on a server, regardless of the environment it is running in.
\setcounter{numberedCntB}{\theenumi}
\end{enumerate}


\begin{enumerate}
\setcounter{enumi}{\thenumberedCntB}
\item Many organizations deploy private registry servers in their 
internal/external application development/deployment environment. 
Although Dockers are considered the standardized method for 
micro-services deployment, playing an important role in cloud computing 
emerging fields such as service meshes, it is understood that container 
security is the main concern and adoption barrier for many companies. 
\setcounter{numberedCntB}{\theenumi}
\end{enumerate}


\underline{AIM OF THE RESEARCH}



\begin{enumerate}
\setcounter{enumi}{\thenumberedCntB}
\item The aim of this project is to survey on container security and 
solutions which will help to understand container security requirements 
and obtain a clearer picture of possible vulnerabilities and attacks. 
\setcounter{numberedCntB}{\theenumi}
\end{enumerate}


\underline{SCOPE OF THE RESEARCH}



4.\ \ \ \ I have identified that the research should cover and not 
limited to the undermentioned points related to security requirements 
within the host-container threat landscape.

a.\ \ \ \ Protecting a container from applications inside it.

b.\ \ \ \ Inter-container protection.

c.\ \ \ \ Protecting the host from containers.

d.\ \ \ \ Protecting containers from a malicious or semi-honest host.



\underline{METHODOLOGY}

5.\ \ \ \ This research will be carried out in three phases as depicted 
below in order to propose with the suitable security measures for Docker 
Image Security.



\newcounter{numberedCntC}
\begin{enumerate}
\item Survey the available literature on container security 
\item Analyze the threat perception
\item Find solutions for identified threat perception
\setcounter{numberedCntC}{\theenumi}
\end{enumerate}


\underline{CONCLUSION}



6.\ \ \ \ Containers are important for the future of cloud computing. 
Micro-services and containers are closely related, where containers are 
considered the standardized way for micro-service deployment. Containers 
are important for the emerging field of service meshes that relies on 
micro-services, too.



7.\ \ \ \ However, one of the primary adoption barriers to container 
widespread deployment is the security issues they face. Therefor this 
research work is attempted to fill this gap by looking at the literature 
and identifying the main threats which are due to image, registry, 
orchestration, container, side channels, and host OS risks.

\end{document}
